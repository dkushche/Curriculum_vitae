\section{Projects}

\hypertarget{proj_rhl}{\subsection{\href{https://github.com/dkushche/RooTHazardLab}{RooTHazardLab}}}
\begin{itemize}
    \item Project description:
    \begin{itemize}
        \item A private Kubernetes lab is hosted on the Yggdrasil network and is accessible via Alfis DNS.
              The core value is the ability to get home projects and tools up and running faster and
              more efficiently, and share them without the need for 'white' IP addresses and commercial DNS
              providers.
    \end{itemize}
    \item Type: Personal
    \item Responsibilities:
    \begin{itemize}
        \item Definition concept of lab: main objective and instruments
        \item Investigation of existing tools and methods to help achieve the goal
        \item Building couple of prototypes
        \item Create and Deploy Lab on existing hardware
    \end{itemize}
    \item Tools \& Technologies:
    \begin{itemize}
        \item Kubernetes
        \item Helm
        \item Ansible
        \item Vagrant
        \item Nginx proxy(ingress)
        \item Python
        \item Go
        \item Jenkins
        \item Yggdrasil
        \item NFS
        \item MetalLB
        \item Clair
        \item git-crypt
        \item Docker Registry
        \item Dockerd
        \item containerd
    \end{itemize}
\end{itemize}
\newpage

\hypertarget{proj_com4}{\subsection{Commercial 4}}
\begin{itemize}
    \item Project description:
    \begin{itemize}
        \item The client is an IoT connection provider offering internet connectivity for any smart product,
              automatically and securely, right out of the box.
    \end{itemize}
    \item Type: Commercial
    \item Responsibilities:
    \begin{itemize}
        \item Building test system infrastructure
              (HTTP(s) servers, Netlink, procfs, sysfs, DNS server mocks).
        \item Improvement of OTA updates for Linux and FreeRTOS devices.
        \item Interaction with wifi driver using vendor API.
        \item Tasks management, colleagues support, participation in the discussions.
    \end{itemize}
    \item Tools \& Technologies:
    \begin{itemize}
        \item Python
        \item Bash
        \item C
        \item Linux
        \item FreeRTOS
        \item Docker
        \item Make
        \item Jenkins
        \item Pytest
        \item Netlink
        \item dpkg
        \item initd
        \item systemd
        \item mcuboot
    \end{itemize}
\end{itemize}
\newpage

\hypertarget{proj_com3}{\subsection{Commercial 3}}
\begin{itemize}
    \item Project description:
    \begin{itemize}
        \item The service coordinates workers and managers and records all data in the
              cloud for further analytics and workflow improvements in smart warehouses,
              manufacturing facilities, field services, and in a variety of other rugged work
              environments.
    \end{itemize}
    \item Type: Commercial
    \item Responsibilities:
    \begin{itemize}
        \item Updating wpa\_supplicant for an industrial wearable device.
        \item Fixing WiFi connection delay issue.
    \end{itemize}
    \item Tools \& Technologies:
    \begin{itemize}
        \item Android
        \item ADB
        \item C
        \item wpa\_supplicant
        \item netlink
        \item make
        \item repo
        \item git
    \end{itemize}
\end{itemize}
\newpage

\hypertarget{proj_com2}{\subsection{Commercial 2}}
\begin{itemize}
    \item Project description:
    \begin{itemize}
        \item The client had a custom router for which he needed firmware. The main task
              was to create a universal firmware with a user-friendly UI, which makes it
              possible possible to configure networks like on advanced routers, as well as to be a
              station for IoT devices.
    \end{itemize}
    \item Type: Commercial
    \item Responsibilities:
    \begin{itemize}
        \item Improving custom WiFi router firmware
        \item Adding user-friendly UI for network policies manipulation:
              isolation (adding VLANs for L2, firewall settings for L3)
        \item Blocking Internet resources(black/white listing)
        \item Captive portal setup
        \item Wifi management
    \end{itemize}
    \item Tools \& Technologies:
    \begin{itemize}
        \item OpenWRT
        \item Lua
        \item LuCI
        \item ip
        \item iptables
        \item brctl
        \item iw
        \item wpa\_supplicant
        \item dnsmasq
        \item CoovaChilli
        \item Squid transparent proxy
        \item git
        \item das u-boot
    \end{itemize}
\end{itemize}
\newpage

\hypertarget{proj_loxotron}{\subsection{\href{https://github.com/dkushche/Loxotron}{Loxotron Casino}}}
\begin{itemize}
    \item Project description:
    \begin{itemize}
        \item A project in which I supported and trained a team of schoolchildren to create their
              own modern web application.
    \end{itemize}
    \item Type: Charity
    \item Responsibilities:
    \begin{itemize}
        \item All Project Management from initiation to the closing phase
        \item CI/CD configuration
        \item Support and training of students
        \item Meetup scheduling and participation
    \end{itemize}
    \item Tools \& Technologies:
    \begin{itemize}
        \item Github Projects
        \item Github Actions
        \item Git
        \item Yarn
        \item Express JS
        \item MongoDB
        \item ReactJS
        \item TypeScript
        \item Webpack
        \item Docker-compose
        \item Ansible
        \item Nginx proxy
        \item Figma
    \end{itemize}
\end{itemize}
\newpage

\hypertarget{proj_com1}{\subsection{Commercial 1}}
\begin{itemize}
    \item Project description:
    \begin{itemize}
        \item The client had a project that used unsafe functions from libc, such as strcpy. It
              created security risks and potential overflows. The main objective was to find
              a and fix all unsafe calls.
    \end{itemize}
    \item Type: Commercial
    \item Responsibilities:
    \begin{itemize}
        \item Securing libc functions calls in existed software.
    \end{itemize}
    \item Tools \& Technologies:
    \begin{itemize}
        \item C
        \item Linux
        \item Redmine
        \item GitLab
    \end{itemize}
\end{itemize}
\newpage
